\subsection{Discusi'on}
En la figura 1 se ve la cantidad de instrucciones realizadas por el algoritmo en funcion de la densidad del grafo. Las pruebas realizadas fueron sobre grafos de 20 nodos y variando la densidad entre 0 y 100. Para comentar, esta prueba tardo 1 hora 10 minutos en una maquina con un Intel Core 2 Duo 1.86Ghz con 2 GB RAM DDR2 1000Mhz. Se tomaron 5 muestras por cada variacion de densidad para tener un resultado un poco mas acertado.
A primera vista se observa como la cantidad de instrucciones es muy grande y tiende a seguir subiendo, recien estabilizandose cuando la densidad va llegando a 100\%.
Un comentario sobre este grafico:
Si bien se ve que en el caso que son completos los grafos, y se tomaron 5 muestras, estos son iguales observacionalmente, no estructuralmente. O sea que la lista de ejes que llega desde el parametro de entrada no siempre esta en el mismo orden y eso varia la cantidad de instrucciones a realizar.

En la figura 2, el grafico muestra la cantidad de instrucciones realizadas, pero en funcion de los nodos del grafo. Aca se ve claramente que la complejidad se cumple, ya que es exponencial y crece rapidamente. Las pruebas fueron realizadas fijando la densidad en 50\% y variando los nodos entre 0 y 20. No se utilizaron mas nodos, ya que la prueba tardaria mas de 3 horas.
