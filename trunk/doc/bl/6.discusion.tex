\subsection{Discusi'on}
Como se puede ver en el primer gr'afico, a partir de cierto punto, mientras ambos par'ametros aumentan el puntaje disminuye dr'asticamente. Pero 'esta acumulaci'on de la importancia cerca de los par'ametros no nos sorprendi'o. Es de esperar que, al sacar y agregar peque'nas cantidades de nodos, se obtenga mayor flexibilidad para generar vecinos y de esa forma conseguir un mayor espectro de posibles vecinos con los cuales probar. Lo que s'i nos sorprendi'o fue que las mismas pruebas fueron corridas con las pruebas utilizando al algoritmo goloso como soluci'on inicial y el resultado fue el mismo: se tend'ia a mejorar el puntaje con par'ametros cercanos al (4,7). Creemos que se debe a que a nuestro algoritmo goloso no lo quisimos hacer minimal, por lo que en cierta medida puede generar soluciones \emph{similares} al algoritmo \emph{naive}, ya que el 'ultimo brinda recubrimientos de gran tama'no.

Los mejores par'ametros fueron\footnote{Para mayor informaci'on sobre los mejores puntajes, dirigirse al ap'endice correspondiente.}: (4,7), (5,7) y (6,7), ley'endose 'estas tuplas como (porcentaje cuantos agrego, porcentaje cuantos saco). Preferimos tomar al menor de ellos como el mejor, y la decisi'on se bas'o principalmente en lo dicho anteriormente: nos parece que menores valores brindan flexibilidad a las soluciones, por lo que podr'ian comportarse mejor. Adem'as, considerando el rango posible de valores, podemos apreciar que valores peque'nos son realmente los ganadores y los mayores generan una meseta en el gr'afico de puntajes. Por ende, el (4,7) es nuestro mejor par.

En el segundo y tercer gr'afico ya utilizamos el par'ametro calculado anteriormente para averiguar la cantidad de instrucciones que cuesta cada ejecuci'on utilizando como soluci'on inicial la \emph{naive} y la golosa.

El primero de los dos utiliza grafos todos con la misma cantidad de nodos pero densidad (porcentaje de la cantidad total de posibles ejes) variable. Notar que $m$ fue dividido por la cantidad m'axima de ejes posibles $\left(\frac{n(n-1)}{2}\right)$ siendo en 'este caso $n=40$) del grafo para obtener la densidad y as'i ser posible plasmar los valores en un gr'afico que se encuentra en funci'on de la densidad. Como se puede apreciar, la cota de la complejidad es v'alida. Pero adem'as se puede apreciar otro hecho importante: la cantidad de instrucciones totales aumenta considerablemente cuando se elige como soluci'on inicial a la golosa por sobre la \emph{naive}. Y era de esperar debido a que la complejidad del primero es mayor que la del segundo.

El segundo de los gr'aficos de instrucciones fija la densidad de los grafos (50\%) y var'ia la cantidad de nodos. La curva te'orica dibujada se baso en que $m$ sea el mayor valor posible en el grafo (es decir, cuando la cantidad de nodos es m'axima: $50*49/2 \approx 600$) y es por eso que nos pareci'o, en principio, tan exagerada la cota. Por otro lado, la complejidad te'orica est'a basada en casos que son pr'acticamente imposibles en la realidad. Por ejemplo, las $n$ llamadas a MejorVecino explicado en la complejidad del algoritmo principal, BusquedaLocal (entre otros). Igualmente no nos preocupa porque en el peor caso es verdad que puede llegar a suceder, asique no se la puede considerar una "mala cota". Con respecto a la comparacion entre las soluciones iniciales, aqu'i se refleja nuevamente el costo superior de la soluci'on golosa por sobre el de la \emph{naive}.

Los dos 'ultimos gr'aficos intentan mostrar la bondad de las soluciones de la b'usqueda local con ambas soluciones iniciales, bas'andonos nuevamente en el mejor par obtenido anteriormente. La primer comparaci'on nos pareci'o que arroj'o resultados favorables ya que, en promedio, la mayor diferencia en el recubrimiento es de uno o nung'un nodo. Notar que a medida que aumenta la densidad del grafo, los errores disminuyen. Sin embargo, en el que realiza comparaciones en funci'on de la cantidad de nodos no se encuentra patr'on alguno (por lo menos para la cantidad de muestras que pudimos computar). 'Esto nos parece extra'no pero la 'unica explicaci'on disponible es que trabajamos con grafos aleatorios, y si quisi'esemos encontrar detalles a 'este nivel deber'iamos analizar grafo por grafo.

Tampoco fueron encontrados patrones respecto a la comparaci'on entre ambas soluciones iniciales ya que mantienen valores similares. Se podr'ia decir que el goloso tiene una cantidad un tanto menor de diferencias con respecto a la \emph{naive}, pero los resultados obtenidos no arrojan pruebas contundentes que aseguren que la primera es mejor que la segunda.
