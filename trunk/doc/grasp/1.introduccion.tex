\subsection{Introducci'on}
La heur'istica GRASP se basa en utilizar en cierta manera las dos utilizadas anteriormente en el presente trabajo. El algoritmo consiste en utilizar B'usqueda Local pero con una soluci'on inicial brindada por un algoritmo goloso con un factor de azar. La soluci'on que brinda se guarda para futuras comparaciones y ser'a actualizado en aquellos casos donde la nueva soluci'on sea mejor que la anterior.

El motivo por el cual se parte de una soluci'on golosa azaroza es porque, al ser GRASP un algoritmo que solamente termina cuando llega a su l'imite de iteraciones maximas o durante unas iteraciones predeterminadas no hubo cambios. Si la solucion fuese la golosa tradicional, siempre partiria de la misma, y siempre llegaria a lo mismo, terminando el algoritmo en los pasos dichos por la cantidad de iteraciones sin cambios.

Es por esto que se agrega un parametro que determina el porcentaje de permisividad para el goloso. Este lo que hace es, tomando el grado del nodo con mas cantidad de vecinos, utiliza solo ese porcentaje del grado. Por ejemplo, si el grado es 10 y el porcentaje de permisividad es del 40\%, entrar'ian a la lista de candidatos todos los que tienen 4 o mas vecinos.

Una vez que tenemos esta soluci'on, se le aplica b'usqueda local, para mejorar la soluci'on, a partir de par'ametros que dicen cuantos nodos agregar y sacar en cada iteracion (mejor explicado en Busqueda Local). 'Estos par'ametros fueron seleccionados luego de varias pruebas donde se le asign'o puntaje a cada par de par'ametros, dependiendo del tama'no de la soluci'on. A menor tama'no, m'as puntaje. Se corri'o varias veces con muchos algoritmos y los par'ametros utilizados son los que lograron mayor puntaje en las pruebas.
