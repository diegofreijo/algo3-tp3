\subsection{Introducci'on}
Un algoritmo goloso es aquel que soluciona un problema mediante la busqueda del optimo local en cada etapa de ejecucion esperando tener al optimo global del
problema.

El algoritmo goloso consta de varias partes:

\begin{itemize}
\item  Un conjunto de candidatos, desde donde se va a formar la solucion
\item  Una funcion de seleccion, que elige al mejor candidato para agregar a la solucion
\item  Una funcion de fiabilidad, que dice si el candidato contribuye a la solucion
\item Una funcion objetivo, que le asigna un valor a la solucion o a la parcial
\item Una funcion solucion, que dice si se llego a una solucion completa

\end{itemize}

Los problemas para los cuales el algoritmo goloso funciona mejor son aquellos que cumplen con las siguientes propiedades:

\begin{itemize}
\item Greedy Choice Property

Esto quiere decir que el problema permite que podamos buscar siempre una solucion optima local, por su naturaleza. Este algoritmo iterativamente crea una 
solucion golosa despues de la otra, haciendo que el problema se reduzca. Un algoritmo goloso nunca reconsidera sus elecciones.

\item  Optimal Substructure 

Una subestructura optima existe si una optima solucion al problema contiene optimas soluciones a sus sub-problemas.

\end{itemize}
