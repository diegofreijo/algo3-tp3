\subsection{Introducci'on}
Los algoritmos de busqueda local se caracterizan por definir, dada una posible soluci'on del problema, soluciones \emph{vecinas}. Dos soluciones se consideran vecinas si ellas son \emph{parecidas} bajo alg'un critero. Definida la funci'on de vecindad, el algoritmo lo que hace es generar una solucion inicial (ya sea utilizando otras heur'isticas, t'ecnicas algor'itmicas o soluciones \emph{ad-hoc}) y proseguir a enumerar sus vecinos. De all'i, y bajo alg'un criterio, se selecciona alguna que se considere \emph{mejor} a fines del problema a resolver. Para ello se define la funci'on objetivo. 'Esta asigna un valor num'erico a cada posible soluci'on de forma tal que, si una soluci'on $s$ es mejor que otra $t$ bajo el criterio que se est'a utilizando, entonces $s$ deber'a tener menor funci'on objetivo (el resultado de aplicar dicha funci'on a $s$ sera menor que 'aquel donde se aplique a $t$). Por lo tanto, el algoritmo podr'a elegir f'acilmente un vecino mejor simplemente evaluando 'esta funci'on. Notar que el criterio de decisi'on tambi'en deber'a incluir alguna forma de elegir una sola soluci'on si es que dos o m'as soluciones vecinas poseen la menor funci'on objetivo. Luego se reinicia el procedimiento pero tomando 'esta vez como soluci'on inicial a la elegida anteriormente como mejor vecina. El algoritmo finaliza cuando no existe una mejor soluci'on que la actual.

Notar que la heur'istica (tal y como por definici'on de heur'istica deber'ia ser) no es exacta, es decir que no brinda (necesariamente) la mejor soluci'on al problema. 'Esto sucede cuando el algoritmo encuentra un m'inimo local de la funci'on objetivo en lugar de encontrar el m'inimo global. Por ello es que es importante elegir una buena soluci'on inicial (una que se encuentre alejada de m'inimos locales) y un buen criterio de selecci'on de mejores vecinos. Pero, obviamente, 'esto es dif'icil de conseguir en la pr'actica.

A pesar de la inseguridad de la soluci'on que brinda, la heur'istica puede resultar muy 'util en comparaci'on con un algoritmo exacto si se utiliza en problemas cuya soluci'on m'as r'apida conocida hasta el momento es exponencial. Y se debe principalmente a la complejidad polinomial que posee y a la existencia de par'ametros, lo cual permite ajustar el algoritmo a las necesidades de cada contexto de uso.
