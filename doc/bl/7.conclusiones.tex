\subsection{Conclusiones}
Para finalizar el an'alisis de la heur'istica, hay dos puntos que son los que m'as nos sorprendieron:
\begin{itemize}
\item La mejora en las soluciones que pueden causar las vecindades

En comparaci'on a otra implementaci'on realizada con anterioridad de la funci'on de vecindad que recorr'ia menos vecinos, los mejores par'ametros pasaron de (0,1) a (4,7). Y los puntajes fueron mejores. Creemos que, aunque la complejidad del algoritmo aument'o a ra'iz de 'esta nueva implementaci'on, no deja de ser polinomial y arroja mejores resultados.

\item Las pocas diferencias con el algoritmo exacto

Es verdad que 'este es un punto que depende netamente del contexto de uso que se le dar'a al algoritmo y a la soluci'on que produzca, pero que ande en errores de la d'ecima parte del total nos parece un punto importante a favor de la heur'istica. Es posible que s'olo se d'e 'este fen'omeno en grafos peque'nos pero, repitiendo, de haber podido hubi'esemos hecho muchas pruebas m'as.

\item La superioridad de la soluci'on inicial \emph{naive}

En un principio cre'imos que agregar una soluci'on con mayor inteligencia podr'ia llevarnos a mejores resultados, pero no fue as'i. El utilizar la soluci'on inicial golosa lo 'unico que caus'o fue que la cantidad de instrucciones aumente considerablemente y las mejoras no son convincentes. Notar adem'as\footnote{Ver ap'endice de puntajes.} la cercan'ia de los mejores par'ametros obtenidos en ambos casos. Pero el mejor puntaje del goloso es muy inferior al mejor del \emph{naive}. Por eso creemos que la soluci'on \emph{naive} funciona mejor para nuestro algoritmo. 

\end{itemize}
