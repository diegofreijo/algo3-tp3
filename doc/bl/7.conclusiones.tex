\subsection{Conclusiones}
Para finalizar el an'alisis de la heur'istica, hay dos puntos que son los que m'as nos sorprendieron:
\begin{itemize}
\item Lo peque'nos que son los mejores par'ametros

Como se dijo antes, cre'imos que 'estos iban a ser m'as elevados, no muy grandes pero tampoco tan peque'nos. Consideramos que se podr'a deber a la forma en la cual elegimos los vecinos: existe mayor libertad para elegir vecinos mientras m'as chicos sean los cambios. Capaz que 'esto favoreci'o a que el algoritmo pudiese encontrar vecinos que, de haber hecho cambios m'as grandes, no hubiese podido. Adem'as la soluci'on naive, al ser de tama'no elevado, ayuda a 'este fen'omeno porque brinda mayores posibilidades para quitar nodos. Nos hubiese gustado correr el algoritmo con mayores valores pero el poder de c'omputo que ten'iamos daba para lo que se hizo y nada mas...

\item Las pocas diferencias con el algoritmo exacto

Es verdad que 'este es un punto que depende netamente del contexto de uso que se le dar'a al algoritmo y a la soluci'on que produzca, pero que ande en errores de la d'ecima parte del total nos parece un punto importante a favor de la heur'istica. Es posible que s'olo se d'e 'este fen'omeno en grafos peque'nos pero, repitiendo, de haber podido hubi'esemos hecho muchas pruebas m'as.
\end{itemize}
