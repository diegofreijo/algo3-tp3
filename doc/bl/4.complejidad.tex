\subsection{An'alisis de complejidad}
\subsubsection{SolucionInicial}
'Este algoritmo recorre todos los ejes ($m$ veces) y para cada uno verifica si existe alguno de sus extremos en la posible soluci'on (como puede llegar a ser todos los nodos, $n$). Por lo tanto la complejidad es $O(nm)$. 

\subsubsection{MejorVecino}
La primer acci'on realizada por el algoritmo que no tiene complejidad constante es la selecci'on de los nodos a agregar. 'Esta es del orden de $n$ (cantidad de nodos del grafo) en el peor caso ya que debe ir comparando para cada uno de los nodos si est'a o no en la soluci'on.

Luego viene el bucle exterior, que va recorriendo cada nodo de la soluci'on. Por lo que se ejecuta $t$ veces, lo cual equivale a O($n$) ya que $n$ es el m'aximo valor posible para $t$.

El segundo bucle se ejecuta tantas veces como \emph{offsets} posibles hallan. 'Esta cantidad esta dominada por la ecuaci'on

$$(cs - 1) * o + cs \leq t - i$$

considerando que $o$ es el offset e $i$ es la posici'on del nodo actual. Por lo tanto el offset m'aximo $om$ ser'a

$$om = (t - i - cs) / (cs - 1)$$

redondeado para abajo. Notar que 'este valor ser'a mayor cuando se est'e al comienzo de la lista ($i=0$),  y cuando la cantidad de nodos que saco tambi'en sea m'inima ($cs=2$, porque de ser 1 no existir'ia el offset ya que se toma solo a $i$) y cuando el tama'no de la soluci'on de entrada sea m'axima ($t=n$). Por lo que se ejecutar'a O($n$) veces.

El tercer y cuarto bucle son similares a los anteriores, y ya que la lista de nodos a poner puede tambi'en ser $n$, las complejidades son las mismas.

Las verificaciones de si es o no un recubrimiento son O($m$) y devolverlos se realizan en tiempo constantes, por lo que se considera a todas las operaciones de los bucles en O($m$).

Por lo tanto, la complejidad total del algoritmo es O($n^4m$)

\subsection{BusquedaLocal}
El algoritmo lo primero que realiza es generar la soluci'on $naive$, lo cual le cuesta $O(nm)$. Luego ejecuta MejorVecino hasta encontrar una mejor soluci'on. Dado que el bucle se ejecuta si hay una mejora en la soluci'on, y que 'esta no puede tener mas de $n$ nodos, el caso en donde m'as veces iterar'a ser'a cuando la soluci'on naive arroje un resultado de longitud $n$ y en cada iteraci'on se disminuya la respuesta en uno, hasta alcanzar el m'inimo en un conjunto con un s'olo nodo. En 'este caso se har'ian $n$ iteraciones, por lo que la complejidad ser'ia

$$O(nm + n(n^4m)) = O(n^5m)$$
