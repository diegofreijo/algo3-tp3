\subsection{An'alisis de complejidad}
La complejidad del exacto es la siguiente. Al comienzo del algoritmo, chequea que la solucion que le entra si es recubrimiento. Esto es $O(m)$. Luego, realiza $2^n$ recursiones, devolviendo cada vez el menor recubrimiento logrado.
Este se consigue realizando dos veces la recursion, una vez con un nodo mas y otro sin ese nodo, sacando el mismo de la lista de nodos.
Tomando un �rbol de decisi�n, cada nodo es un nivel del �rbol, y sus dos hijos son la llamada recursiva a la funci�n agregando este nodo y la llamada recursiva sin agregar el nodo. Por lo tanto, tenemos un �rbol binario completo de $n$ niveles, teniendo $2^{n}$ nodos, donde cada nodo es una llamada a la funci�n ExactoRecursivo. 
Con n tama�o del problema, la f�rmula para la complejidad recursiva queda de la siguiente manera:

$T(n) = 2 * T(n-1) + c = 4 * T(n-2) + 3 * c = ... = 2^{n-1} * T(1) + (2^{n-1}-1) * c = O(2^{n} + 2^{n} *c )$

$c$ es el costo que tiene cada llamada recursiva, el cual es $m$. Las llamadas son a lo sumo $O(d)$, siendo $d$ el grado de un nodo. Esto se puede acotar por n, ya que el grado de un nodo no puede ser mayor a la cantidad de nodos del grafo.

En conclusi�n, la complejidad de la funci�n es $O(2^{n} + m * 2^{n})$ = $O(m * 2^{n})$, donde n es la cantidad de nodos del gr�fico original.
