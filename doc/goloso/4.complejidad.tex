\subsection{An'alisis de complejidad}
La complejidad del algoritmo goloso es bastante simple. Este algoritmo lo primero que realiza es sacar los aislados del grafo, estos son los nodos que no tienen vecinos.
Por consiguiente, estos nodos no modifican al vertex cover ya que no recubren ningun eje.
Esto lo realiza en $O(n)$, recorriendo toda la lista de nodos y sacando los que no se van a usar.
Luego, lo que realiza el algoritmo es un ciclo en donde se le va agregando a la solucion el nodo que mayor cantidad de vecinos tenga. Este ciclo se repite
hasta que la solucion ya es un recubrimiento. Este ciclo se repite, en peor caso, n veces. Dentro del ciclo, la funcion que busca al nodo con mas vecino, tiene complejidad $O(n)$ y la funcion que verifica si es recubrimiento es $O(m)$, ya que itera sobre la lista de ejes.
Las demas operaciones son en $O(1)$. 
La complejidad resultante es de $O(n(n+m)+n) \rightarrow O(n^2+n*m+n) \rightarrow O(n^2)$ 
