\subsection{Introducci'on}
La heur'istica GRASP se basa en utilizar en cierta manera las dos utilizadas anteriormente en el presente trabajo. El algoritmo consiste en utilizar B'usqueda Local pero con una soluci'on inicial brindada por un algoritmo goloso con un factor de azar. La soluci'on que brinda se guarda para futuras comparaciones y ser'a actualizado en aquellos casos donde la nueva soluci'on sea mejor que la anterior.
El motivo por el cual se parte de una solucion golosa azaroza es porque, al ser GRASP un algoritmo que solamente termina cuando llega a su limite de iteraciones maximas o durante unas iteraciones predeterminadas no hubo cambios. Si la solucion fuese la golosa tradicional, siempre partiria de la misma, y siempre llegaria a lo mismo, terminando el algoritmo en los pasos dichos por la cantidad de iteraciones sin cambios.
Es por esto que se agrega un parametro que determina el porcentaje de permisividad para el goloso. Este lo que hace es, tomando el grado del nodo con mas cantidad de vecinos, utiliza solo ese porcentaje del grado. Por ej. si el grado es 10 y el porcentaje de permisividad es del 40\%, entrarian a la lista de candidatos todos los que tienen 4 o mas vecinos.
Una vez que tenemos esta solucion, se le aplica busqueda local, para mejorar la solucion, a partir de parametros que dicen cuantos nodos agregar y sacar en cada iteracion (mejor explicado en Busqueda Local). Estos parametros fueron seleccionados luego de varias pruebas donde se le asigno puntaje a cada par de parametros, dependiendo del tama�o de la solucion. A menor tama�o, mas puntaje. Se corrio varias veces con muchos algoritmos y los parametros utilizados son los que lograron mayor puntaje en las pruebas.
