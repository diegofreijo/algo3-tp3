\subsection{Conclusiones}
Las conclusiones que tenemos para GRASP es que la nueva l'ogica que se agreg'o gener'o cambios 'utiles. Aunque se increment'o dr'asticamente la cantidad de instrucciones, las diferencias con el exacto fueron menores. No creemos que 'esto marque al algoritmo mejor que los anteriores, ya que en aplicaciones de tiempo real 'este incremento el las instrucciones puede conllevar a acciones indeseadas (como la p'erdida en la continuidad de la ejecuci'on, conllevando a una insatisfacci'on del usuario).

Pero 'esto tambi'en nos indica que la b'usqueda local puede llegar a alcanzar mejores soluciones que con los vecinos que recorre, porque en GRASP - gracias a que se ejecuta el mismo algoritmo varias veces pero cambiando la soluci'on inicial - se verifican en total mas cantidad de vecinos. Y si la ejecuci'on repetitiva de la b'usqueda local lleva a cada vez mejores algoritmos, es porque la busqueda local efectivamente no recorre todos los vecinos posibles (cosa que es deseada ya que de no ser as'i estar'iamos hablando de un algoritmo exacto mas que una heur'istica).
