\section{Introducci'on}
El problema planteado para resolver es el llamado "Vertex Cover". Este consiste en encontrar un subconjunto de nodos tal que para todos los ejes del grafo,
por lo menos uno de los nodos del subconjunto sea extremo de este.
Para dar un ejemplo, en un grafo asi

\imagen{img/grafoIntro.png}{6}{Grafo Ejemplo}

los nodos {1,3,5,6} son un vertex cover, pero tambien lo es el {2,4,5}.

Nuestro objetivo es, por medio de heuristicas (descriptas mas adelante), tratar de llegar al minimo vertex cover. Este es el minimo subconjunto que cumpla
con la condicion.

Este problema pertenece a la familia de los problemas NP-Completos en la teoria de complejidades y es uno de los 21 problemas NP-Completos de Karp. Los
problemas NP-Completos son los mas dificiles de la familia de los NP ("non-deterministic polynomial time").

Algunos ejemplos en donde se ve la aplicacion de esta resolucion de problema, serian:

\begin{itemize}
\item Se desean poner camaras de seguridad en el coqueto barrio Parque, tal que sea la minima cantidad de camaras y que estas vigilen todas las cuadras.
Para esto se contrata a un grupo de computadores, que modelan el problema poniendo a las esquinas como nodos y a las cuadras como ejes entre esos nodos.
Al aplicarle la solucion, se les podra dar una respuesta satisfactoria a los vecinos, y asi garantizar su seguridad, tanto de sus casas como de sus bolsillos.
\item Un sistema de audio moderno, tiene, para no escatimar, muchos parlantes. Cada persona, por su poco oido, necesita que el sonido llegue desde uno solo de ellos, no es necesario mas. Se necesita llevarse algunos parlantes para otra sala. Al aplicarle la solucion, se utilizan a las personas como nodos y al sonido de cada parlante como eje. Se quitan aquellos parlantes tal que todas sus ondas de sonido lleguen a personas que ya estan escuchando por otro.

\end{itemize}}

Al no tener solucion en tiempo polinomico (es decir de la forma $n^k$), en un grafo medianamente chico, el tiempo de ejecucion puede llevar mucho tiempo.
Por ejemplo, un grafo de 50 nodos, va a realizar, en el caso del algoritmo exacto mas conocido (o sea de complejidad $O(2^n)$), $2^{50}$ operaciones, es decir
1.125.899.906.842.624 operaciones. Aproximando una cantidad de operaciones por segundo de 1.000.000, tardaria mas de 357020 siglos en dar una solucion exacta.
Por mas que se tenga una maquina 100 veces mas rapida, igual tardaria mucho, o sea mas de 3570 siglos.	

Por este motivo, se implementaron diferentes heuristicas (algoritmo que provee una combinacion de buena solucion y rapidez de ejecucion, pero sin garantizar,
que la respuesta sea correcta).

La primera de ellas fue la heuristica golosa. Un algoritmo goloso es aquel que, para resolver el problema, hace una seleccion local optima en cada iteracion
del mismo, con el objetivo de buscar un optimo global.
En el caso del vertex cover, la funcion de seleccion optima local del goloso es buscar aquel nodo que tenga mayor cantidad de vecinos, con lo que se espera
que poniendo estos nodos, se llegue mas rapido a recubrir todo el grafo.
En el caso de un grafo rueda, por ejemplo, este algoritmo cumple con ser el minimo vertex cover, pero en los algoritmos aleatorios no siempre cumple con
ese requisito (ver mas adelante en los graficos).


La segunda heuristica implementada fue busqueda local. Esta consiste en, a partir de una solucion inicial, moverse a traves de diferentes soluciones hasta
que no aparezca una solucion mejor en ninguno de sus vecinos o hasta que se cumpla una condicion de parada (puede ser por tiempo o por cantidad de iteraciones
realizadas).
Graficando el problema como una funcion, esta heuristica busca un maximo o minimo local (dependiendo lo que se quiera realizar), que se espera que sea tambien
el global, aunque no siempre coincide.


La tercera heuristica usada fue GRASP (Greedy randomized adaptive search procedure). Esta heuristica consiste en iteraciones hechas a partir de  construcciones
de una solucion inicial tomada de un algoritmo goloso random y una consiguiente secuencia de mejoras a partir de de la busqueda local.


A continuacion se explica con mayor detalle cada uno de los algoritmos implementados, asi como tambien su pseudocodigo, complejidad y conclusiones.
Para finalizar, se evaluaran las 3 alternativas y se comparara con el exacto para observar cual fue la que mas se acerco a la solucion optima.
